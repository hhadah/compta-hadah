\begin{table}
\centering
\caption{Summary Statistics for County-Year Data}\label{tab01:sum}
\centering
\begin{tabular}[t]{lrrrrr}
\toprule
  & Mean & SD & Min & Max & N\\
\midrule
Total Suicide Rate (per 100k) & \num{13.64} & \num{14.14} & \num{0.00} & \num{575.54} & 80,313\\
Firearm Suicide Rate (per 100k) & \num{8.45} & \num{11.00} & \num{0.00} & \num{431.65} & 80,313\\
Men's Firearm Suicide Rate (per 100k) & \num{14.86} & \num{19.92} & \num{0.00} & \num{677.97} & 80,313\\
Firearm Suicide Rate Aged 55+ (per 100k) & \num{24.99} & \num{221.42} & \num{0.00} & \num{33,333.33} & 80,313\\
White Individuals' Firearm Suicide Rate (per 100k) & \num{16.11} & \num{97.85} & \num{0.00} & \num{9,000.00} & 80,313\\
Non-Firearm Suicide Rate (per 100k) & \num{5.19} & \num{7.55} & \num{0.00} & \num{358.42} & 80,313\\
Female Population Share & \num{0.50} & \num{0.02} & \num{0.28} & \num{0.62} & 80,313\\
\bottomrule
\end{tabular}
\vspace{0.2em}
\begin{minipage}{0.95\linewidth}
\footnotesize\emph{Notes}: Data come from the National Vital Statistics System (NVSS), 1959--2019. Suicide rates are expressed per 100{,}000 population. ``Female Population Share'', ``College Educated'', and ``Below Poverty Line'' are proportions. SD denotes standard deviation. N is the number of county--year observations (80{,}313).
\end{minipage}
\end{table}
