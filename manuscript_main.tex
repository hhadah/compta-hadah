%%%%%%%%%%%%%%%%%%%%%%%%%%%%%%%%%%%
% Main Text
%%%%%%%%%%%%%%%%%%%%%%%%%%%%%%%%%%%
\section{Introduction}

Suicide claims nearly one million lives worldwide each year and is often an impulsive act \autocite{lewieckiSuicideGunsPublic2013}. In the United States, firearms---especially handguns---are involved in more than half of all gun deaths and roughly half of suicides, and the economic cost of suicides is large \autocite{statisticsNationalVitalStatistics2007,greenbergEconomicBurdenAdults2015,greenbergEconomicBurdenAdults2021}. Since gunshots are highly lethal and require little planning, policies that introduce a barrier between purchase and possession may be uniquely positioned to save lives. Waiting period laws create such a barrier, giving suicidal intent time before committing suicide using firearms. Therefore, it is important to causally identify whether waiting periods are effective in preventing firearm suicide.

Emerging research in economics and public health further reinforces that suicide is not solely a function of long-standing mental illness, but is highly responsive to acute shocks and the availability of means. Economic hardship—such as job loss, income volatility, or relative status decline—has been shown to significantly increase suicide rates \autocite{breuer2014unemployment, christian2019income, daly2013relative}. The results support the idea that the implementation of policy measures aimed at impulsive moments, such as waiting periods for firearm purchases, can significantly influence the outcomes by disrupting critical time frames that could otherwise lead to deadly actions.

Moreover, impulsive-aggressive behavior has consistently been identified as a risk factor for suicide across the life course, with adolescents and young adults being particularly vulnerable \autocite{anestis2014impulsivity, mcgirr2008impulsive}. Many suicide attempts are driven by transient states of hopelessness and distress rather than long-term ideation. Consequently, restricting immediate access to lethal means like firearms provides a crucial which suicidal urges may subside or intervention may occur.


A substantial body of evidence links easy access to firearms with an increased risk of suicide. International comparisons find strong correlations between household gun ownership and suicide rates, with no signs that people simply switch to other means when guns are less available \autocite{killiasInternationalCorrelationsGun1993}. In the United States, research shows that unloaded guns and separate ammunition storage are associated with markedly lower odds of suicide by youth \autocite{grossmanGunStoragePractices2005}. These findings align with clinical observations that many suicide attempts suddenly arise during moments of acute psychological distress \autocite{lewieckiSuicideGunsPublic2013}.

Cross-national policy evaluations reinforce the value of restricting rapid access to firearms. Following the tightening of gun laws in 1992, New Zealand saw a 46\% decrease in firearm suicides among the general population and a 66\% decrease among individuals aged 15–24\autocite{beautrais2006firearms}. The 1996 Australian National Firearms Agreement, which combined large gun buybacks with stricter licensing, has also been associated with subsequent declines in firearm suicide \autocite{bakerGunLawsSudden2007}.\footnote{The Australian National Firearms Agreement's buyback provision was a mandatory government purchase program in 1996-1997 that collected approximately 650,000 prohibited firearms from civilians at market value, funded by a temporary Medicare levy. Participation was compulsory, with criminal penalties for non-compliance after the amnesty period.} In the United States, the laws on firearm removal based on risk ('red flag') enacted in Connecticut and Indiana were followed by measurable reductions in statewide suicide rates \autocite{kivisto2018effects}.

Firearm-related injuries and suicides among youth remain a significant public health concern in the United States. Research has shown that mental health diagnoses often precede youth suicides, underscoring the need for earlier identification and intervention strategies within healthcare systems \autocite{chaudhary2024youth}. Concurrently, the financial burden of firearm injuries—both fatal and nonfatal—has been substantial, with estimates indicating billions in annual costs for healthcare and lost productivity \autocite{miller2024costs}. These challenges are compounded by persistent trends of firearm-related harm in children and adolescents, emphasizing the urgent need for multifaceted prevention and harm-reduction approaches \autocite{lee2022firearm, kaufman2021epidemiologic}. This urgency is magnified by psychological research emphasizing the impulsive nature of many youth suicides, where firearm access dramatically increases the risk of fatal outcomes \autocite{anestis2014impulsivity, mcgirr2008impulsive}.


Efforts to reduce firearm-related injuries must also focus on storage practices and perceptions of accessibility. \textcite{miller2025firearm} highlight the critical role of secure firearm storage in reducing suicide risk, particularly in households with adolescents. However, firearm storage practices vary widely, with older adults and parents often underestimating the extent to which firearms are accessible to youth \autocite{carter2022firearm, hastings2025parental}. Even when parents report using safe storage methods, teens may still perceive firearms as accessible, suggesting that education and behavioral interventions need to account for both parent and youth perspectives \autocite{hastings2025parental}. A combination of public health education, policy enforcement, and clinical interventions is essential to address the intertwined issues of access, perception, and injury prevention.

Despite the existence of many studies on gun control policies and suicide rates, there is little research on the causal effect of gun control policies on suicide rates. The empirical literature notably lacks causal analyses of US waiting-period laws on firearm suicides. he existing literature is generally limited to individual states, short periods following legislative actions, or relies on broad national metrics, thereby complicating efforts to disentangle causal estimates from simultaneous external factors such as the opioid crisis or economic fluctuations, which also play crucial roles in shaping suicide patterns. To our knowledge, this will consequently be the first paper aiming to causally estimate the effect of waiting periods on suicides. 

We link six decades of county-level mortality data from the National Vital Statistics System with the longitudinal RAND State Firearm Law Database, which codes the exact timing of every waiting period statute enacted between 1813 and 2015 \autocite{statisticsNationalVitalStatistics2007,cherneyDevelopmentRANDState2022}. We also create a crosswalk that would allow full use of county-level mortality data. The staggered adoption of waiting-period laws across states creates a natural experiment that allows us to compare suicide trajectories in treated, not-yet-treated, and never-treated states.

We use the recent developments in the difference-in-differences literature to implement an event study design that exploits both cross-state and within-state overtime variation in exposure to waiting periods. The specification includes county-fixed effects to absorb time-invariant local confounders and year-fixed effects to control for common shocks. 

We find a sharp and sustained decline in firearm suicides among men. Waiting periods among the broad population, adults aged 55+, and White individuals did not show statistically significant effects, but the confidence intervals mostly lay in the negative territory. The lack of imprecise estimates does not mean that waiting periods did not cause a decrease in firearm suicides. On average, waiting periods reduce county firearm-suicide rates by 0.5 deaths per 100,000 population—roughly a 7 percent drop relative to baseline, but it is statistically insignificant. The effects are greater for men and adults 55 years and older, two groups that account for the majority of firearm suicides. Crucially, we detect no significant change in non-firearm suicides among adults 55 years and older, men, and white individuals, consistent with the view that waiting periods curb fatalities by delaying access to a uniquely lethal method.

The remainder of this paper is structured as follows. Section \ref{sec:data} discusses the current data. We introduce an empirical approach in Section \ref{sec:emp-model}, then summarize and discuss the results in Section \ref{sec:results}. The conclusions of the study are presented in Section \ref{sec:conc}. 

\section{Data} \label{sec:data}

Two main data sources are used. To measure the effect of waiting period on firearm suicides, we use mortality data from the National Vital Statistics System (NVSS). We also use RAND's state firearm law dataset for waiting periods.

\subsection*{Firearm suicide data}

To measure the effect of waiting periods on firearm suicides in the United States, we use mortality data from the National Vital Statistics System (NVSS) covering the years 1959 to 2019 \autocite{statisticsNationalVitalStatistics2007}. Our outcome of interest is the firearm suicide rate, defined as the number of firearm suicides per 100,000 population in each county and year. We specifically use the Multiple Cause of Death files, which provide detailed causes of death for each death recorded in the US at the county level using ICD-10 codes. The ICD-10 codes allow for the identification of specific causes of death, including suicide. Suicides are further broken down into several categories, allowing for a more detailed analysis of different types of suicide, including those involving firearms.\footnote{The ICD-10 codes used to defined underlying causes of death due to suicide were X60-X84 (intentional self-harm), and Y87.0 (Sequelae of intentional self-harm). X60 to X69 correspond to intentional self-poisoning, while X70 to X84 correspond to intentional self-harm by other and unspecified means, including drowning, hanging, strangulation, and suffocation, smoke, sharp object, etc. Suicide by firearms was categorized using three specific codes: X72 (intentional self-harm by handgun discharge), X73 (intentional self-harm by rifle, shotgun and larger firearm discharge), and X74 (intentional self-harm by other and unspecified firearm discharge).} In addition, the data include a range of socioeconomic characteristics of the deceased, such as age, sex, race, marital status, and education level. For each death, we also have information on the county of occurrence, county of residence, and county population size.\footnote{Using information on counties that merged and split from \textcite{forstall1994population, bailey2015war}, we recombine all counties that split or merge after 1959 to produce a cross-walk of Multiple Cause of Death files. This consequently allowed us to have a balanced panel of all counties from 1959 to 2019.}

\subsection*{State firearm law}

The second dataset used is the RAND State Firearm Law Database, developed as part of the Gun Policy in America initiative in 2016. It is a longitudinal database that tracks all gun laws by state from 1813 to 2015 \autocite{cherneyDevelopmentRANDState2022}. The database covers various categories of gun laws, including background check requirements for handguns and long guns, firearm sales restrictions, minimum age requirements, and the presence of waiting periods, defined as the time a seller must wait between the purchase and delivery of a firearm.

The RAND database enables us to construct the treatment variable based on the presence of a waiting period in a given state and year. Several states are considered ``never-takers,'' meaning that they never implemented waiting periods for firearm purchases. These states include Colorado, Delaware, Iowa, Massachusetts, Michigan, Missouri, Nebraska, Nevada, New York, North Carolina, Ohio, South Carolina, Utah, and Virginia. Among the states that implemented waiting periods, two scenarios arise: (1) states with a single policy transition, meaning they switched from no waiting period to having the one only once, and (2) states with multiple transitions between implementation and non-implementation. States with a single policy transition include California, the District of Columbia (DC), Florida, Hawaii, Illinois, Maryland, Minnesota, Mississippi, New Jersey, Rhode Island, and Washington. The remaining states did not experience policy transitions. We present different cohorts of states in Table \ref{tab:tab-01} and a map of all states with different treatment sequences in Figure \ref{fig:211-county_year_panel_vieww}.

\subsection*{Sample construction}

% We construct two samples to use for our analysis.

The sample is composed of states that experience either one policy transition from no-treatment to treatment, or none at all. This yields a sample of 23 states to be retained for the final analysis: 10 states are part of the treatment group, meaning that those states adopted a waiting period before 1959 or between 1959 and 2019, and 13 are part of the control group, meaning that those states never adopted a waiting period for firearms purchases. In terms of counties, the final sample is composed of 1,289 counties. A total of 870 counties as part of the 'never-takers', six counties adopted a waiting period prior to 1959, and 413 counties adopted a waiting period between 1959 and 2019. Figures \ref{fig:192-staggarred_sample_panel_view} and \ref{fig:194-staggarred_sample_panel_view} provide a graphical overview of the states and counties included in our final sample, along with their adoption status of the waiting period policies. Figure \ref{fig:192-staggarred_sample_panel_view} displays the staggered adoption of waiting periods, while Figure \ref{fig:194-staggarred_sample_panel_view} presents the corresponding county count. For the purpose of our analysis, the comparison group is composed of never-takers and yet-to-be-treated. For population and demographic information, we combined our final dataset with data from the Decennial US Census \autocite{uscensus2020}. 

We also construct a second sample of states that move out of treatment. The sample is composed of states that experience either one policy transition from treatment to no-treatment, or always treated. This yields a sample of 23 states to be retained for the final analysis: 13 states are part of the treatment group, meaning that those states adopted a waiting period before 1959 or between 1959 and 2019, and 13 are part of the control group, meaning that those states never adopted a waiting period for firearms purchases.

\subsection*{Outcome variable}

The main outcome variable is the suicide rate by firearm per 100,000 people. We identify suicides related to firearms using ICD-10 codes for the underlying causes of death: X72 (intentional self-harm by handgun discharge), X73 (intentional self-harm by rifle, shotgun, and larger firearm discharge) and X74 (intentional self-harm by other and unspecified firearm discharge).

We then calculated county-level suicide rates by aggregating individual-level mortality data for each year and county. Specifically, we sum the number of firearm suicides in each county-year and divide by the corresponding county population, multiplying by 100,000 to calculate the suicide rate by firearm in a given county and year. You can find trends in firearm and non-firearm suicide rates from 1959 to 2019 in Figures \ref{fig:suicide_firearm_byyear_popcont} and \ref{fig:06-suicide_byyear_popcont}.  There is a substantial heterogeneity in firearm suicide rates by treatment status (Figure \ref{fig:01-suicide_firearm_byyearbystat}), with consistently lower rates observed in states that implemented waiting period policies, as well as marked demographic disparities (Figure \ref{fig:221-suicide_firearm_byyear_raceagesex}), with particularly elevated rates among men and people over 55+ years of age. 

\subsection*{Summary statistics}

The sample statistics are presented in Table \ref{tab01:sum}. Counties that adopted a mandatory waiting period tend to have lower total suicide rates per 100,000 people, overall firearm suicide rates, men, older adults (age 55 and older) and white individual firearm suicide than counties that never adopted a mandatory waiting period. In addition, counties that adopted a mandatory waiting period tend to have a similar share of the female population, college educated, and tend to have a lower proportion of their population living below the poverty line. 

\section{Empirical strategy} \label{sec:emp-model}

We exploit two sources of variation in exposure to the waiting period to assess its effect on firearm suicide rates.  The first source of variation comes from the state-level adoption of firearm purchase waiting periods, which allows us to compare between counties that have adopted waiting periods and those that have not. The second source of variation is based on the timing of firearm suicides, with some suicides that occurred before and after the adoption of waiting periods within the adopting states. The first source of variation compares outcomes between counties that have adopted waiting periods and those that have not. The second source of variation is based on the timing of suicides, comparing suicides that occurred before and after the adoption of waiting periods within adopting states. The identifying assumption is that absent the adoption of waiting periods, counties that adopted waiting periods and states that never adopted them would have followed similar trends in suicide rates by firearms (parallel trends assumption) and that no individual does not anticipate treatment (no anticipation assumption). \\

We are using the approach from \textcite{callaway2021difference} to estimate how waiting periods affect firearm suicide rates in different counties. Our event study looks at county i, in state s, at time t using the following equation:

\begin{equation} \label{eq:1}
y_{ist} = \sum_{l = -K}^{L} \beta_l \mathbf{1}\{ t - E_s = l \} + \theta_i + \lambda_t + \varepsilon_{ist}
\end{equation}

where the outcome variable \textit{$y_{ist}$} is the number of suicides by firearms per 100,000 people in county \textit{i}, in state \textit{s}, at time \textit{t}. $E_s$ is the period of time in which state \textit{s} adopted a waiting period. $\mathbf{1}\{ t - E_s = l \}$ is an indicator variable equal to 1 when time \textit{t} is \textit{l} periods away from the adoption of waiting periods in state \textit{s}. For example, Florida adopted waiting periods on January 10\textsuperscript{th} 1991, therefore $t-E_s$ would give you the number of periods away from the year 1991. If $t=1988$, Florida would be three years away from waiting periods, which means that $t-E_s$ would be equal to three. All regressions include county-fixed effects $\theta_i$ and time period fixed effects $\lambda_t$. We used counties not yet treated and never treated as a control group. All standard errors are clustered bootstrapped at the state level.\\

The coefficients of interest is $ \beta_l $. The coefficients for values of $l < 0$ provide an estimate of the effect of the waiting period on our outcome of interest before the adoption of the waiting period. We can also test for parallel trends and without an anticipation assumption using coefficients for values of $l < 0$. If the coefficient $\beta_{-1}$ is statistically insignificant, it indicates that the assumption of no anticipation holds. If coefficients $ \beta_l $ for values of $l < 0$ are statistically insignificant, then the assumption of parallel trends holds. On the other hand, the coefficient for values of $l \geq 0$ captures the post-treatment effect of waiting periods on suicide rates by firearms.\\

To evaluate whether waiting period laws simply shift methods of suicide rather than reducing overall suicide rates, we analyze their impact on nonfirearm suicide rates. If waiting periods genuinely decrease total suicides, we would expect no significant change in non-firearm suicide methods. Using the same analytical approach (equation \ref{eq:1}) with non-firearm suicide rates as our outcome variable, we test this relationship. The absence of any significant effect on nonfirearm suicides would strengthen the validity of our core assumption that waiting periods specifically prevent firearm-related suicides rather than causing method substitution. 

For the analysis examining states that move out of treatment (i.e., states that repeal their waiting period laws), the treatment group consists of states that transition from having a waiting period to not having one, while the control group consists of states that consistently maintain waiting periods throughout the study period. This reverse treatment design allows us to examine whether the removal of waiting periods leads to increases in firearm suicide rates, providing additional evidence for the causal effect of these policies.

\section{Results} \label{sec:results} 

\subsection*{The Effects of Waiting Period Adoption on Firearm Suicide Rates}

\textbf{Overall Effect on Firearm Suicides.} The results of the estimation of the first specification are shown in Figure \ref{fig:firearm_suicide_DID_overall}. The figure shows the estimates 10 years before the adoption of waiting periods and 10 years after. For the 10 years before the adoption of waiting periods, we show points estimates and their associated 95\% confidence intervals, which correspond to pre-periods. Those pre-treatment estimates can be used to assess parallel trends assumption. For 10 years after adoption, it presents the point estimates and their associated confidence intervals 95\% for the post-adoption periods. These estimates correspond to the treatment effects. The figure shows a statistically significant decrease in the suicide rate by firearm in the periods following the adoption of waiting periods. In fact, adopting waiting periods reduces the suicide rate by firearms by about 0.47 deaths per 100,000 people, but the results are statistically insignificant. Although the negative points estimates are also statistically insignificant, the majority of the confidence intervals are in the negative side. 

\textbf{Effect on men.} Figure \ref{fig:firearm_suicide_DID_men} presents the results of the event study estimates when restricting the sample to men only. In the post-treatment period, the average effect shows a reduction in firearm suicides among men by 1.3 deaths per 100,000 after the adoption of waiting periods. This effect is statistically significant at 90\% confidence level. 

\textbf{Effect on adults aged 55 and older. } The results of the estimation of the first specification for adults 55 years and older are shown in Figure \ref{fig:firearm_suicide_DID_older}. Estimates before treatment are not statistically significant from zero, supporting the assumption of parallel trends. After the adoption of waiting periods, the average effect shows a statistically significant reduction in firearm suicides for adults 55 years of age and older by 26.3 deaths per 100,000 (p-value = 0.18). 

\textbf{Effect on white individuals.} Figure \ref{fig:firearm_suicide_DID_white} shows the causal effect of waiting period laws on firearm suicides among white individuals. The pretreatment supports the assumption of parallel trends. The average effect shows a statistically significant reduction in firearm suicides in white individuals by 21.6 deaths per 100,000 (p-value = 0.09), which represents a significant decrease in the suicide rate by firearm among this population.

\textbf{Effect on other causes of suicide} Figures \ref{fig:firearm_suicide_DID_other}-\ref{fig:firearm_suicide_DID_other_white} shows the causal effect of waiting period laws on other causes of suicide (i.e., all causes of suicide excluding suicides by firearm). Once again, the pretreatment estimates support the assumption of parallel trends. Adopting waiting periods is associated with an increase of 0.40 deaths per 100,000 (p-value = 0.135). However, this effect is not statistically significant from zero, confirming that people did not commit suicide using other methods after a state adopts waiting periods. The same holds when estimating the effect of waiting periods on other causes of suicide among men and adults ages 55+, and a significant decrease among White individuals.

\subsection*{The Effects of Waiting Period Repeal on Firearm Suicide Rates}
\textbf{Overall Effect of Repeal on firearm suicides.} The results of the estimation of the first specification are shown in Figure \ref{fig:firearm_suicide_DID_overall-out}. The figure shows the estimates 5 years before the repeal of waiting periods and 5 years after. For the 5 years before the repeal of waiting periods, we show point estimates and their associated 95\% confidence intervals, which correspond to pre-periods. Those pre-treatment estimates can be used to assess parallel trends assumption. For 5 years after repeal, it presents the point estimates and their associated 95\% confidence intervals for the post-adoption periods. These estimates correspond to the treatment effects of repealing waiting periods on firearm suicides. The figure shows a statistically insignificant decrease in the suicide rate by firearm in the periods following the repeal of waiting periods. Although most of the point estimates are negative, they are all imprecisely estimated.

\textbf{Effect on men.} Figure \ref{fig:firearm_suicide_DID_men-out} presents the results of the event study estimates when restricting the sample to men only. In the post-repeal period, the average effect shows a statistically insignificant reduction in firearm suicides among men. The point estimates are imprecisely estimated, consequently indicating that we cannot rule out either substantial increases or decreases in firearm suicide rates following the repeal of waiting periods.

\textbf{Effect on adults aged 55 and older. } The results of the estimation of the first specification for adults 55 years and older are shown in Figure \ref{fig:firearm_suicide_DID_older-out}. Estimates before repeal are not statistically significant from zero, supporting the assumption of parallel trends. After the repeal of waiting periods, the average effect shows a statistically insignificant point estimates. This suggests that we cannot rule out substantial increases or decreases in firearm suicide rates following the repeal of waiting periods.

\textbf{Effect on white individuals.} The results of the estimation of the first specification for White individuals are shown in Figure \ref{fig:firearm_suicide_DID_white-out}. Estimates before repeal are statistically insignificant, supporting the assumption of parallel trends. Although the point estimate at $t=-1$ is statistically significant, it is unlikely that this is a violation of the no anticipation assumption since the estimate is positive rather than negative, which would be expected if individuals were anticipating the repeal by increasing firearm suicides before the policy change. After the repeal of waiting periods, the average effect shows a statistically insignificant point estimates. This suggests that we cannot rule out substantial increases or decreases in firearm suicide rates following the repeal of waiting periods.

\textbf{Effect on other causes of suicide} Figures \ref{fig:firearm_suicide_DID_other-out}-\ref{fig:firearm_suicide_DID_other_white-out} shows the causal effect of the repeal of the waiting period on other causes of suicide (i.e., all causes of suicide excluding suicides by firearm). Once again, the pretreatment estimates support the assumption of parallel trends. The repeal of waiting periods is associated with a decrease of 0.33 deaths per 100,000 (p-value = 0.142). The same holds when estimating the effect of the repeal of waiting periods on other causes of suicide among men and adults ages 55+, and White individuals. 

\section{Conclusion} \label{sec:conc}

This study provides the most extensive evidence to date that waiting periods for firearm purchases may be an effective population-level suicide prevention tool. Leveraging decades of county-by-year data and the full historical record of state gun laws, we show that waiting-period laws lower firearm-suicide rates among men without inducing offsetting increases in other methods. Although the effects for the broader population, older adults and White individuals were not statistically significant, confidence intervals consistently fell in the negative range, suggesting potential protective effects that warrant further investigation and consideration for a broader implementation.

Policymakers debating gun violence interventions often face a trade-off between political feasibility and measurable impact. Waiting periods appear to offer an unusually favorable balance: they impose minimal costs on lawful purchasers while generating sizable, precisely estimated reductions in mortality. Future work should examine heterogeneous effects in urban and rural contexts, explore interactions with red flag and safe storage laws, and evaluate whether analogous 'cooling-off' requirements for other lethal means (for example, large quantities of prescription opioids) could yield similar life-saving dividends.

